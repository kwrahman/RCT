\need{Inference}

\begin{itemize}
    \item If you are treating entire ``clusters'' of observations (e.g., entire schools or firms), but your data are at a lower level (e.g., students or workers), cluster your standard errors at the level of the treatment.
    \item Small number of clusters? No problem! Try methods in \cite{Cameron} or \cite{hagemann}.
    \item If you are evaluating a large number of outcomes, consider adjusting your inference for the fact that you are testing multiple hypotheses (\cite{Anderson}).
    \item Want to test a ``sharp'' null hypothesis, that the treatment effect is 0 for all participants (rather than the effect is 0 on average)? Try randomization inference! See, e.g., \cite{young}.
\end{itemize}

%\divider\smallskip

%\itemtag{other}
%\itemtag{more}
%\itemtag{used before (would do)}
%\itemtag{twesors}


\need{How the Pros Do It}
\begin{itemize}
\item Ask other researchers what they think the outcome of the RCT will be, using \url{https://socialscienceprediction.org/}. 
\item How much external validity does your experiment have? Consider examining this using methods such as \cite{kowalski}.
    \end{itemize}



\need{Rating}

\risk{Difficulty}{4}
\risk{Fun}{5}
\risk{Validity}{5}


\need{Make it Sizzle}
\begin{itemize}
    \item If you have the sample to support multiple treatment arms, try out variations of your treatment to better understand mechanisms.
\end{itemize}




